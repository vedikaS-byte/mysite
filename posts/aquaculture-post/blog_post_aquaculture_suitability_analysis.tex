% Options for packages loaded elsewhere
\PassOptionsToPackage{unicode}{hyperref}
\PassOptionsToPackage{hyphens}{url}
\PassOptionsToPackage{dvipsnames,svgnames,x11names}{xcolor}
%
\documentclass[
  letterpaper,
  DIV=11,
  numbers=noendperiod]{scrartcl}

\usepackage{amsmath,amssymb}
\usepackage{iftex}
\ifPDFTeX
  \usepackage[T1]{fontenc}
  \usepackage[utf8]{inputenc}
  \usepackage{textcomp} % provide euro and other symbols
\else % if luatex or xetex
  \usepackage{unicode-math}
  \defaultfontfeatures{Scale=MatchLowercase}
  \defaultfontfeatures[\rmfamily]{Ligatures=TeX,Scale=1}
\fi
\usepackage{lmodern}
\ifPDFTeX\else  
    % xetex/luatex font selection
\fi
% Use upquote if available, for straight quotes in verbatim environments
\IfFileExists{upquote.sty}{\usepackage{upquote}}{}
\IfFileExists{microtype.sty}{% use microtype if available
  \usepackage[]{microtype}
  \UseMicrotypeSet[protrusion]{basicmath} % disable protrusion for tt fonts
}{}
\makeatletter
\@ifundefined{KOMAClassName}{% if non-KOMA class
  \IfFileExists{parskip.sty}{%
    \usepackage{parskip}
  }{% else
    \setlength{\parindent}{0pt}
    \setlength{\parskip}{6pt plus 2pt minus 1pt}}
}{% if KOMA class
  \KOMAoptions{parskip=half}}
\makeatother
\usepackage{xcolor}
\setlength{\emergencystretch}{3em} % prevent overfull lines
\setcounter{secnumdepth}{-\maxdimen} % remove section numbering
% Make \paragraph and \subparagraph free-standing
\ifx\paragraph\undefined\else
  \let\oldparagraph\paragraph
  \renewcommand{\paragraph}[1]{\oldparagraph{#1}\mbox{}}
\fi
\ifx\subparagraph\undefined\else
  \let\oldsubparagraph\subparagraph
  \renewcommand{\subparagraph}[1]{\oldsubparagraph{#1}\mbox{}}
\fi

\usepackage{color}
\usepackage{fancyvrb}
\newcommand{\VerbBar}{|}
\newcommand{\VERB}{\Verb[commandchars=\\\{\}]}
\DefineVerbatimEnvironment{Highlighting}{Verbatim}{commandchars=\\\{\}}
% Add ',fontsize=\small' for more characters per line
\usepackage{framed}
\definecolor{shadecolor}{RGB}{241,243,245}
\newenvironment{Shaded}{\begin{snugshade}}{\end{snugshade}}
\newcommand{\AlertTok}[1]{\textcolor[rgb]{0.68,0.00,0.00}{#1}}
\newcommand{\AnnotationTok}[1]{\textcolor[rgb]{0.37,0.37,0.37}{#1}}
\newcommand{\AttributeTok}[1]{\textcolor[rgb]{0.40,0.45,0.13}{#1}}
\newcommand{\BaseNTok}[1]{\textcolor[rgb]{0.68,0.00,0.00}{#1}}
\newcommand{\BuiltInTok}[1]{\textcolor[rgb]{0.00,0.23,0.31}{#1}}
\newcommand{\CharTok}[1]{\textcolor[rgb]{0.13,0.47,0.30}{#1}}
\newcommand{\CommentTok}[1]{\textcolor[rgb]{0.37,0.37,0.37}{#1}}
\newcommand{\CommentVarTok}[1]{\textcolor[rgb]{0.37,0.37,0.37}{\textit{#1}}}
\newcommand{\ConstantTok}[1]{\textcolor[rgb]{0.56,0.35,0.01}{#1}}
\newcommand{\ControlFlowTok}[1]{\textcolor[rgb]{0.00,0.23,0.31}{#1}}
\newcommand{\DataTypeTok}[1]{\textcolor[rgb]{0.68,0.00,0.00}{#1}}
\newcommand{\DecValTok}[1]{\textcolor[rgb]{0.68,0.00,0.00}{#1}}
\newcommand{\DocumentationTok}[1]{\textcolor[rgb]{0.37,0.37,0.37}{\textit{#1}}}
\newcommand{\ErrorTok}[1]{\textcolor[rgb]{0.68,0.00,0.00}{#1}}
\newcommand{\ExtensionTok}[1]{\textcolor[rgb]{0.00,0.23,0.31}{#1}}
\newcommand{\FloatTok}[1]{\textcolor[rgb]{0.68,0.00,0.00}{#1}}
\newcommand{\FunctionTok}[1]{\textcolor[rgb]{0.28,0.35,0.67}{#1}}
\newcommand{\ImportTok}[1]{\textcolor[rgb]{0.00,0.46,0.62}{#1}}
\newcommand{\InformationTok}[1]{\textcolor[rgb]{0.37,0.37,0.37}{#1}}
\newcommand{\KeywordTok}[1]{\textcolor[rgb]{0.00,0.23,0.31}{#1}}
\newcommand{\NormalTok}[1]{\textcolor[rgb]{0.00,0.23,0.31}{#1}}
\newcommand{\OperatorTok}[1]{\textcolor[rgb]{0.37,0.37,0.37}{#1}}
\newcommand{\OtherTok}[1]{\textcolor[rgb]{0.00,0.23,0.31}{#1}}
\newcommand{\PreprocessorTok}[1]{\textcolor[rgb]{0.68,0.00,0.00}{#1}}
\newcommand{\RegionMarkerTok}[1]{\textcolor[rgb]{0.00,0.23,0.31}{#1}}
\newcommand{\SpecialCharTok}[1]{\textcolor[rgb]{0.37,0.37,0.37}{#1}}
\newcommand{\SpecialStringTok}[1]{\textcolor[rgb]{0.13,0.47,0.30}{#1}}
\newcommand{\StringTok}[1]{\textcolor[rgb]{0.13,0.47,0.30}{#1}}
\newcommand{\VariableTok}[1]{\textcolor[rgb]{0.07,0.07,0.07}{#1}}
\newcommand{\VerbatimStringTok}[1]{\textcolor[rgb]{0.13,0.47,0.30}{#1}}
\newcommand{\WarningTok}[1]{\textcolor[rgb]{0.37,0.37,0.37}{\textit{#1}}}

\providecommand{\tightlist}{%
  \setlength{\itemsep}{0pt}\setlength{\parskip}{0pt}}\usepackage{longtable,booktabs,array}
\usepackage{calc} % for calculating minipage widths
% Correct order of tables after \paragraph or \subparagraph
\usepackage{etoolbox}
\makeatletter
\patchcmd\longtable{\par}{\if@noskipsec\mbox{}\fi\par}{}{}
\makeatother
% Allow footnotes in longtable head/foot
\IfFileExists{footnotehyper.sty}{\usepackage{footnotehyper}}{\usepackage{footnote}}
\makesavenoteenv{longtable}
\usepackage{graphicx}
\makeatletter
\def\maxwidth{\ifdim\Gin@nat@width>\linewidth\linewidth\else\Gin@nat@width\fi}
\def\maxheight{\ifdim\Gin@nat@height>\textheight\textheight\else\Gin@nat@height\fi}
\makeatother
% Scale images if necessary, so that they will not overflow the page
% margins by default, and it is still possible to overwrite the defaults
% using explicit options in \includegraphics[width, height, ...]{}
\setkeys{Gin}{width=\maxwidth,height=\maxheight,keepaspectratio}
% Set default figure placement to htbp
\makeatletter
\def\fps@figure{htbp}
\makeatother

\usepackage{booktabs}
\usepackage{longtable}
\usepackage{array}
\usepackage{multirow}
\usepackage{wrapfig}
\usepackage{float}
\usepackage{colortbl}
\usepackage{pdflscape}
\usepackage{tabu}
\usepackage{threeparttable}
\usepackage{threeparttablex}
\usepackage[normalem]{ulem}
\usepackage{makecell}
\usepackage{xcolor}
\KOMAoption{captions}{tableheading}
\makeatletter
\makeatother
\makeatletter
\makeatother
\makeatletter
\@ifpackageloaded{caption}{}{\usepackage{caption}}
\AtBeginDocument{%
\ifdefined\contentsname
  \renewcommand*\contentsname{Table of contents}
\else
  \newcommand\contentsname{Table of contents}
\fi
\ifdefined\listfigurename
  \renewcommand*\listfigurename{List of Figures}
\else
  \newcommand\listfigurename{List of Figures}
\fi
\ifdefined\listtablename
  \renewcommand*\listtablename{List of Tables}
\else
  \newcommand\listtablename{List of Tables}
\fi
\ifdefined\figurename
  \renewcommand*\figurename{Figure}
\else
  \newcommand\figurename{Figure}
\fi
\ifdefined\tablename
  \renewcommand*\tablename{Table}
\else
  \newcommand\tablename{Table}
\fi
}
\@ifpackageloaded{float}{}{\usepackage{float}}
\floatstyle{ruled}
\@ifundefined{c@chapter}{\newfloat{codelisting}{h}{lop}}{\newfloat{codelisting}{h}{lop}[chapter]}
\floatname{codelisting}{Listing}
\newcommand*\listoflistings{\listof{codelisting}{List of Listings}}
\makeatother
\makeatletter
\@ifpackageloaded{caption}{}{\usepackage{caption}}
\@ifpackageloaded{subcaption}{}{\usepackage{subcaption}}
\makeatother
\makeatletter
\@ifpackageloaded{tcolorbox}{}{\usepackage[skins,breakable]{tcolorbox}}
\makeatother
\makeatletter
\@ifundefined{shadecolor}{\definecolor{shadecolor}{rgb}{.97, .97, .97}}
\makeatother
\makeatletter
\makeatother
\makeatletter
\makeatother
\ifLuaTeX
  \usepackage{selnolig}  % disable illegal ligatures
\fi
\IfFileExists{bookmark.sty}{\usepackage{bookmark}}{\usepackage{hyperref}}
\IfFileExists{xurl.sty}{\usepackage{xurl}}{} % add URL line breaks if available
\urlstyle{same} % disable monospaced font for URLs
\hypersetup{
  pdftitle={Ocean Superfarms: Finding the Best Spots to Grow Seafood},
  pdfauthor={Vedika Shirtekar},
  colorlinks=true,
  linkcolor={blue},
  filecolor={Maroon},
  citecolor={Blue},
  urlcolor={Blue},
  pdfcreator={LaTeX via pandoc}}

\title{Ocean Superfarms: Finding the Best Spots to Grow Seafood}
\author{Vedika Shirtekar}
\date{2025-12-06}

\begin{document}
\maketitle
\ifdefined\Shaded\renewenvironment{Shaded}{\begin{tcolorbox}[frame hidden, breakable, sharp corners, enhanced, interior hidden, borderline west={3pt}{0pt}{shadecolor}, boxrule=0pt]}{\end{tcolorbox}}\fi

\hypertarget{ocean-superfarms-finding-the-best-spots-to-grow-seafood}{%
\section{\texorpdfstring{\textbf{Ocean Superfarms: Finding the Best
Spots to Grow
Seafood}}{Ocean Superfarms: Finding the Best Spots to Grow Seafood}}\label{ocean-superfarms-finding-the-best-spots-to-grow-seafood}}

\hypertarget{purpose}{%
\subsection{\texorpdfstring{\textbf{Purpose}}{Purpose}}\label{purpose}}

\textbf{GitHub Repository Link:}
\url{https://github.com/vedikaS-byte/suitability-aquaculture-analysis}

There is a growing global need to sustainably feed expanding human
populations, and one promising solution is marine aquaculture.
Aquaculture is defined as the organized cultivation, feeding,
propagation, and production of aquatic organisms for commercial,
recreational, or public use (Alleway et al., 2018). The aquaculture
industry has significant potential to support food security while also
contributing to conservation and economic development. In California,
aquaculture is highly diverse in its production systems, cultured
species, and final products, accounting for approximately 6\% of the
total value of the U.S. aquaculture industry (Wright et al., 2025).
Wright et al.~identify four main sectors within California aquaculture:
finfish, shellfish, algae, and natural resource agencies. Finfish and
shellfish production together represent more than 70\% of the industry's
total value and primarily supply food for human consumption (Wright et
al., 2025). As a result, commercial aquaculture has become an
increasingly important and profitable source of marine food production
in California.

\begin{figure}

{\centering \includegraphics{images/kelp_forest.png}

}

\caption{\textbf{Figure 1. Kelp forest pictured in Channel Islands
National Park
(\href{https://blog.nature.org/2019/01/21/aquaculture-could-be-conservations-secret-weapon/}{The
Nature Conservancy}).}}

\end{figure}

Oysters and the critically endangered red abalone are two commercially
significant species in California aquaculture. Oysters represent the
largest share of mollusk aquaculture production in the state, followed
by abalone, mussels, and marine clams, which together comprise the
remainder of California's mollusk industry output (Wright et al., 2025).
Although oysters are relatively sturdy animals, they are sensitive to
environmental conditions and are typically confined to sea surface
temperatures (SSTs) between 11-30°C and water depths ranging from 0-70
meters below sea level (Oliver, 2025). The critically endangered red
abalone (\emph{Haliotis rufescens}), a marine gastropod, thrives in
cooler SSTs of 8-18°C and is generally limited to depths of 0-25 meters
below sea level (SeaBase, n.d.). Both species are known to inhabit
waters along the U.S. West Coast. As such, identifying regions where
their suitable environmental ranges overlap within Exclusive Economic
Zones (EEZs) may help evaluate potential areas for sustainable and
profitable aquaculture development.

The purpose of this blog post is to evaluate the suitability of West
Coast Exclusive Economic Zones (EEZs) for developing marine aquaculture
for multiple oyster species and the red abalone. Suitable locations are
identified using species-specific ranges of sea surface temperature
(SST) and ocean depth. This project utilizes GIS applications such as
working with vector and raster data, raster resampling, masking, and map
algebra to analyze environmental conditions. The analysis includes two
main components: (1) creating a final map of suitable oyster aquaculture
areas, and (2) building a function that allows users to generate
suitability maps by EEZ based on selected species, temperature ranges,
and depth limits. The following research question was utilized to
facilitate the analysis:

\textbf{How do changes in species-specific temperature and depth
thresholds affect the identification of suitable aquaculture areas
within West Coast Exclusive Economic Zones (EEZs) for commercially
significant marine species?}

\hypertarget{data-description}{%
\subsection{Data Description}\label{data-description}}

The Bren School of Environmental Science and Management at the
University of California, Santa Barbara provided essential support for
this project, including access to data resources, documentation, and
technical guidance.

\href{https://www.sealifebase.ca/search.php}{SeaLifeBase} is a publicly
available online database that provides information on a wide range of
marine species based on criteria such as commercial importance and
taxonomic group. Each species profile includes details on depth range,
temperature preference, geographic distribution, life history, and
classification. The database is open access and does not require any
downloads to use.

For this analysis, West Coast Exclusive Economic Zone (EEZ) boundaries
were obtained from \href{https://marineregions.org/}{Marineregions.org}
to define maritime boundaries. The
\href{https://www.gebco.net/data_and_products/gridded_bathymetry_data/\#area}{General
Bathymetric Chart of the Oceans (GEBCO)} is a global terrain model that
provides geospatial information, such as elevation, for both ocean and
land surfaces at a 15-arc-second grid resolution (GEBCO, n.d.).
Bathymetry data for this analysis were downloaded from GEBCO as a
\texttt{tif} file. This analysis also used satellite-derived average sea
surface temperature (SST) data from 2008--2012 to characterize mean SST
within each EEZ. Although the GeoTIFF files were provided through course
materials, the original data are publicly available from
\href{https://coralreefwatch.noaa.gov/product/5km/index_5km_ssta.php}{NOAA's
5 km Daily Global Satellite Sea Surface Temperature Anomaly v3.1
product}.

\hypertarget{part-1-map-of-suitable-locations-for-oyster-aquaculture}{%
\subsection{\texorpdfstring{\textbf{Part 1: Map of Suitable Locations
for Oyster
Aquaculture}}{Part 1: Map of Suitable Locations for Oyster Aquaculture}}\label{part-1-map-of-suitable-locations-for-oyster-aquaculture}}

\hypertarget{data-preparation}{%
\subsubsection{\texorpdfstring{\textbf{Data
Preparation}}{Data Preparation}}\label{data-preparation}}

Before beginning my analysis, I loaded the appropriate packages to
properly read, manipulate, and work with geospatial data in formats such
as \texttt{sf}, \texttt{SpatRaster}, and \texttt{SpatVector}. I used the
\texttt{vect()} function to read in the West Coast EEZ shapefile as a
vector layer. I imported the bathymetry dataset using the
\texttt{rast()} function and stored it as a \texttt{SpatRaster} object.
For the sea surface temperature (SST) data, I identified all
\texttt{.tif} files using \texttt{list.files()} and combined them into a
single raster stack with \texttt{rast()} for further analysis.

\begin{Shaded}
\begin{Highlighting}[]
\CommentTok{\# Import packages}
\FunctionTok{library}\NormalTok{(here) }\CommentTok{\# Load "here" to locate and reference files}
\FunctionTok{library}\NormalTok{(tidyverse) }\CommentTok{\# Load the tidyverse" for data cleaning}
\FunctionTok{library}\NormalTok{(sf) }\CommentTok{\# Load "sf" for GIS analysis}
\FunctionTok{library}\NormalTok{(raster) }\CommentTok{\# Load "raster" for accessing raster data types}
\FunctionTok{library}\NormalTok{(ggplot2) }\CommentTok{\# Load "ggplot2" for data visualization}
\FunctionTok{library}\NormalTok{(tmap) }\CommentTok{\# Load "tmap" for functions to create and layer maps}
\FunctionTok{library}\NormalTok{(kableExtra) }\CommentTok{\# Load "kableExtra" for table formatting}
\FunctionTok{library}\NormalTok{(stars) }\CommentTok{\# Load "stars" for integration with "sf"}
\FunctionTok{library}\NormalTok{(terra) }\CommentTok{\# Load "terra" for SpatVector and SpatRaster operations}
\CommentTok{\# knitr::opts\_knit$set(root.dir = here::here("posts", "aquaculture{-}post"))}
\CommentTok{\# getwd()}
\end{Highlighting}
\end{Shaded}

\begin{Shaded}
\begin{Highlighting}[]
\CommentTok{\# West Coast EEZ}
\NormalTok{eez }\OtherTok{\textless{}{-}} \FunctionTok{vect}\NormalTok{(here}\SpecialCharTok{::}\FunctionTok{here}\NormalTok{(}\StringTok{"posts"}\NormalTok{, }\StringTok{"aquaculture{-}post"}\NormalTok{, }\StringTok{"data\_aqua"}\NormalTok{, }\StringTok{"wc\_regions\_clean.shp"}\NormalTok{))}

\CommentTok{\# Bathymetry raster}
\NormalTok{depth }\OtherTok{\textless{}{-}} \FunctionTok{rast}\NormalTok{(here}\SpecialCharTok{::}\FunctionTok{here}\NormalTok{(}\StringTok{"posts"}\NormalTok{, }\StringTok{"aquaculture{-}post"}\NormalTok{,}\StringTok{"data\_aqua"}\NormalTok{, }\StringTok{"depth.tif"}\NormalTok{))}
\end{Highlighting}
\end{Shaded}

\begin{Shaded}
\begin{Highlighting}[]
\CommentTok{\# Create a list of the tiff files for SST}
\NormalTok{sst\_files }\OtherTok{\textless{}{-}} \FunctionTok{list.files}\NormalTok{( }\CommentTok{\# Define file path}
  \AttributeTok{path =} \StringTok{"/Users/vedikashirtekar/Documents/MEDS/mysite/posts/aquaculture{-}post/data\_aqua"}
\NormalTok{,}
  \CommentTok{\# Pattern matches all tiff files starting with   "average\_annual"}
  \AttributeTok{pattern =} \StringTok{"average\_annual.*}\SpecialCharTok{\textbackslash{}\textbackslash{}}\StringTok{.tif$"}\NormalTok{,}
  \AttributeTok{full.names =}\NormalTok{ T }
\NormalTok{)}


\CommentTok{\# Stop running if character(0) returned}
\ControlFlowTok{if}\NormalTok{ (}\FunctionTok{length}\NormalTok{(sst\_files) }\SpecialCharTok{==} \DecValTok{0}\NormalTok{) \{}
  \FunctionTok{stop}\NormalTok{(}\StringTok{"No SST raster files found in directory."}\NormalTok{)}
\NormalTok{\}}

\CommentTok{\# Stack raster files}
\NormalTok{sst }\OtherTok{\textless{}{-}} \FunctionTok{rast}\NormalTok{(sst\_files)}
\end{Highlighting}
\end{Shaded}

It is critical to ensure that the coordinate reference system (CRS) of
all geospatial objects match to avoid misalignment, inaccurate
measurements, and errors during spatial analysis. Here, I used
conditional statements with \texttt{st\_crs()} to detect and resolve any
CRS mismatches between spatial objects and to display a message when
mismatches were found.

\begin{Shaded}
\begin{Highlighting}[]
\CommentTok{\# Create list of spatial objects}
\NormalTok{spatial\_objects }\OtherTok{\textless{}{-}} \FunctionTok{list}\NormalTok{(eez, depth, sst)}

\CommentTok{\# Use eez\textquotesingle{}s CRS as reference}
\NormalTok{ref\_crs }\OtherTok{\textless{}{-}} \FunctionTok{st\_crs}\NormalTok{(spatial\_objects}\SpecialCharTok{$}\NormalTok{eez)}
\end{Highlighting}
\end{Shaded}

\begin{Shaded}
\begin{Highlighting}[]
\CommentTok{\# Check and transform each tile with if/else statements}
\ControlFlowTok{if}\NormalTok{ (}\FunctionTok{st\_crs}\NormalTok{(spatial\_objects}\SpecialCharTok{$}\NormalTok{depth) }\SpecialCharTok{!=}\NormalTok{ ref\_crs) \{}
  \FunctionTok{warning}\NormalTok{(}\StringTok{"depth CRS does not match. }
\StringTok{          Transforming to match eez CRS."}\NormalTok{)}
\NormalTok{  spatial\_objects}\SpecialCharTok{$}\NormalTok{depth }\OtherTok{\textless{}{-}} \FunctionTok{st\_transform}\NormalTok{(spatial\_objects}\SpecialCharTok{$}\NormalTok{depth, ref\_crs)}
\NormalTok{\} }\ControlFlowTok{else}\NormalTok{ \{}
  \FunctionTok{message}\NormalTok{(}\StringTok{"depth CRS already matches eez CRS."}\NormalTok{)}
\NormalTok{\}}
\DocumentationTok{\#\# depth CRS already matches eez CRS.}
\end{Highlighting}
\end{Shaded}

\begin{Shaded}
\begin{Highlighting}[]
\CommentTok{\# Check and transform each tile with if/else statements}
\ControlFlowTok{if}\NormalTok{ (}\FunctionTok{st\_crs}\NormalTok{(spatial\_objects}\SpecialCharTok{$}\NormalTok{sst) }\SpecialCharTok{!=}\NormalTok{ ref\_crs) \{}
  \FunctionTok{warning}\NormalTok{(}\StringTok{"sst CRS does not match. }
\StringTok{          Transforming to match eez CRS."}\NormalTok{)}
\NormalTok{  spatial\_objects}\SpecialCharTok{$}\NormalTok{sst }\OtherTok{\textless{}{-}} \FunctionTok{st\_transform}\NormalTok{(spatial\_objects}\SpecialCharTok{$}\NormalTok{sst, ref\_crs)}
\NormalTok{\} }\ControlFlowTok{else}\NormalTok{ \{}
  \FunctionTok{message}\NormalTok{(}\StringTok{"sst CRS already matches eez CRS."}\NormalTok{)}
\NormalTok{\}}
\DocumentationTok{\#\# sst CRS already matches eez CRS.}
\end{Highlighting}
\end{Shaded}

\hypertarget{data-processing}{%
\subsubsection{\texorpdfstring{\textbf{Data
Processing}}{Data Processing}}\label{data-processing}}

Before combining the SST and depth data, it was necessary to conduct
several preprocessing steps to ensure accurate spatial alignment and
analysis. I first reprojected the depth dataset to match the SST CRS to
ensure proper spatial alignment. The, I created a single raster
representing average SST from 2008--2012 using the \texttt{mean()}
function and converted the average temperature from Kelvin to Celsius by
subtracting 273.15. To prepare the datasets for analysis, I ensured both
rasters matched in terms of a shared CRS, resolution, and extent. Then,
I cropped and resampled (\texttt{resample()}) the depth raster to the
SST boundary to match the resolution of the average SST raster using the
nearest neighbor method.

\begin{Shaded}
\begin{Highlighting}[]
\CommentTok{\# Reproject to match sst CRS}
\NormalTok{depth }\OtherTok{\textless{}{-}} \FunctionTok{project}\NormalTok{(depth, sst)}

\CommentTok{\# Calculate average SST among all rasters}
\NormalTok{avg\_sst }\OtherTok{\textless{}{-}} \FunctionTok{mean}\NormalTok{(sst)}

\CommentTok{\# Update avg\_sst in degrees Celsius}
\NormalTok{avg\_sst }\OtherTok{\textless{}{-}}\NormalTok{ avg\_sst }\SpecialCharTok{{-}} \FloatTok{273.15}

\CommentTok{\# Do the CRS match?}
\FunctionTok{message}\NormalTok{(}\StringTok{"Do the CRS match:"}\NormalTok{, }\FunctionTok{crs}\NormalTok{(avg\_sst) }\SpecialCharTok{==} \FunctionTok{crs}\NormalTok{(depth))}
\DocumentationTok{\#\# Do the CRS match:TRUE}

\CommentTok{\# Do the resolutions match (require resampling)? }
\FunctionTok{message}\NormalTok{(}\StringTok{"Do the resolutions match:"}\NormalTok{, }\FunctionTok{res}\NormalTok{(avg\_sst) }\SpecialCharTok{==} \FunctionTok{res}\NormalTok{(depth))}
\DocumentationTok{\#\# Do the resolutions match:TRUETRUE}

\CommentTok{\# Do the extents match?}
\FunctionTok{message}\NormalTok{(}\StringTok{"Do the extents match: "}\NormalTok{, }\FunctionTok{ext}\NormalTok{(avg\_sst) }\SpecialCharTok{==} \FunctionTok{ext}\NormalTok{(depth))}
\DocumentationTok{\#\# Do the extents match: TRUE}
\end{Highlighting}
\end{Shaded}

\begin{Shaded}
\begin{Highlighting}[]
\CommentTok{\# Use crop() to crop depth to the extent of avg\_sst}
\NormalTok{depth\_sst\_crop }\OtherTok{\textless{}{-}} \FunctionTok{crop}\NormalTok{(depth, avg\_sst)}

\CommentTok{\# Resample with nearest neighbor method}
\NormalTok{depth\_sst\_crop }\OtherTok{\textless{}{-}} \FunctionTok{resample}\NormalTok{(depth\_sst\_crop, avg\_sst, }\AttributeTok{method =} \StringTok{"near"}\NormalTok{)}
\end{Highlighting}
\end{Shaded}

\hypertarget{finding-suitable-locations}{%
\subsubsection{\texorpdfstring{\textbf{Finding suitable
locations}}{Finding suitable locations}}\label{finding-suitable-locations}}

To identify suitable aquaculture locations, I reclassified the average
SST and depth rasters using oyster-specific temperature and depth
thresholds in \texttt{classify()}, where suitable values were assigned a
value of 1 and unsuitable values a value of 0. I then defined a function
to combine both layers by multiplying the reclassified rasters to
identify areas that met \emph{both} environmental requirements. Finally,
I stacked the reclassified SST and depth layers using \texttt{lapp()}
and applied the function to generate a single binary raster showing
overall suitability.

\begin{Shaded}
\begin{Highlighting}[]
\CommentTok{\# Preferred oyster range for SST: 11{-}30°C}

\CommentTok{\# Define reclass matrix for un/suitable SST}
\NormalTok{reclass\_matrix\_sst }\OtherTok{\textless{}{-}} \FunctionTok{matrix}\NormalTok{(}
  \FunctionTok{c}\NormalTok{(}\SpecialCharTok{{-}}\ConstantTok{Inf}\NormalTok{, }\DecValTok{11}\NormalTok{, }\DecValTok{0}\NormalTok{, }\CommentTok{\# Negative infinity (unbounded) to 11 degrees assigned 0}
    \DecValTok{11}\NormalTok{, }\DecValTok{30}\NormalTok{, }\DecValTok{1}\NormalTok{, }\CommentTok{\# 11{-}30 degrees assigned 1}
    \DecValTok{30}\NormalTok{, }\ConstantTok{Inf}\NormalTok{, }\DecValTok{0}\NormalTok{), }\CommentTok{\# 30 to infinity (unbounded) assigned 0}
  \AttributeTok{ncol =} \DecValTok{3}\NormalTok{, }\CommentTok{\# Create three columns}
  \AttributeTok{byrow =}\NormalTok{ T }\CommentTok{\# Fill by row }
\NormalTok{)}

\CommentTok{\# Assign reclassified values to avg\_sst}
\NormalTok{avg\_sst\_reclass }\OtherTok{\textless{}{-}} \FunctionTok{classify}\NormalTok{(avg\_sst, }\AttributeTok{rcl =}\NormalTok{ reclass\_matrix\_sst)}
\end{Highlighting}
\end{Shaded}

\begin{Shaded}
\begin{Highlighting}[]
\CommentTok{\# Preferred oyster range for depth: 0{-}70 meters below sea level}

\CommentTok{\# Define reclass matrix for un/suitable depth}
\NormalTok{reclass\_matrix\_depth }\OtherTok{\textless{}{-}} \FunctionTok{matrix}\NormalTok{(}
  \FunctionTok{c}\NormalTok{(}\SpecialCharTok{{-}}\ConstantTok{Inf}\NormalTok{, }\SpecialCharTok{{-}}\DecValTok{70}\NormalTok{, }\DecValTok{0}\NormalTok{, }\CommentTok{\# {-}70 used for values below sea level}
    \SpecialCharTok{{-}}\DecValTok{70}\NormalTok{, }\DecValTok{0}\NormalTok{, }\DecValTok{1}\NormalTok{, }\CommentTok{\# Suitable values assigned 1 for {-}70{-}0}
    \DecValTok{0}\NormalTok{, }\ConstantTok{Inf}\NormalTok{, }\DecValTok{0}\NormalTok{), }\CommentTok{\# \textgreater{} 0 assigned unsuitable (0)}
  \AttributeTok{ncol =} \DecValTok{3}\NormalTok{,}\CommentTok{\# Create three columns}
  \AttributeTok{byrow =}\NormalTok{ T }\CommentTok{\# Fill by row  }
\NormalTok{)}

\CommentTok{\# Assign reclassified values to depth}
\NormalTok{depth\_reclass }\OtherTok{\textless{}{-}} \FunctionTok{classify}\NormalTok{(depth\_sst\_crop, }\AttributeTok{rcl =}\NormalTok{ reclass\_matrix\_depth)}
\end{Highlighting}
\end{Shaded}

\begin{Shaded}
\begin{Highlighting}[]
\CommentTok{\# Create multiplication function to reference in lapp}
\NormalTok{multiply }\OtherTok{\textless{}{-}} \ControlFlowTok{function}\NormalTok{(x,y)\{ }
\NormalTok{  multi\_raster }\OtherTok{\textless{}{-}}\NormalTok{ x}\SpecialCharTok{*}\NormalTok{y }\CommentTok{\# Raster multiplication across all cells}
  \FunctionTok{return}\NormalTok{(multi\_raster)}
\NormalTok{  \}}
\end{Highlighting}
\end{Shaded}

\begin{Shaded}
\begin{Highlighting}[]
\CommentTok{\# Return suitable (1) and unsuitable (0) cells }
\NormalTok{avg\_sst\_depth }\OtherTok{\textless{}{-}} \FunctionTok{lapp}\NormalTok{(}
  \AttributeTok{x =} \FunctionTok{c}\NormalTok{(avg\_sst\_reclass,depth\_reclass), }\CommentTok{\# Stack rasters}
  \AttributeTok{fun =}\NormalTok{ multiply) }\CommentTok{\# Apply multiplication function }
\end{Highlighting}
\end{Shaded}

\hypertarget{determine-the-most-suitable-eez}{%
\subsubsection{\texorpdfstring{\textbf{Determine the most suitable
EEZ}}{Determine the most suitable EEZ}}\label{determine-the-most-suitable-eez}}

In order to rank EEZ zones by aquaculture potential, it was important to
determine the total suitable area within each EEZ. I calculated the
total suitable area within each zone, then projected the EEZ shapefile
to match the CRS of the suitability raster (\texttt{avg\_sst\_depth}). I
then utilized the \texttt{ifel()} function to identify suitable cells
within the EEZs by reclassifying values of 0 as \texttt{NA} (unsuitable)
and converting all remaining values to 1 (suitable), allowing only
viable areas to be included in the following calculations.

\begin{Shaded}
\begin{Highlighting}[]
\CommentTok{\# Project}
\NormalTok{eez }\OtherTok{\textless{}{-}} \FunctionTok{project}\NormalTok{(eez, avg\_sst\_depth)}

\CommentTok{\# Select suitable areas}
\NormalTok{avg\_sst\_depth\_suitable }\OtherTok{\textless{}{-}} \FunctionTok{ifel}\NormalTok{(avg\_sst\_depth }\SpecialCharTok{==} \DecValTok{0}\NormalTok{, }
                               \ConstantTok{NA}\NormalTok{, }\CommentTok{\# Replace with NA}
                               \DecValTok{1}\NormalTok{) }\CommentTok{\# Otherwise assign "1"}
\end{Highlighting}
\end{Shaded}

Next, I calculated the total suitable aquaculture area (km²) within each
EEZ by masking, calculating cell area, and summarizing suitability by
region. It was necessary to rasterize the \texttt{eez}
\texttt{SpatVector} because vector EEZ polygons are needed to operate in
raster space so area can be summarized per region using raster-based
functions. I rasterized the EEZ polygons using \texttt{rasterize()} such
that each raster cell was labeled by region, permitting area
calculations to be performed in raster space using zonal statistics.
Additionally, I created a mask to only include raster values inside EEZ
regions and exclude suitable locations outside these boundaries. The
output was a raster containing the surface area of cells using
\texttt{cellSize()}. Finally, I used \texttt{zonal()} to sum the area of
suitable cells for each EEZ and joined the results back to the spatial
dataset to visualize the distribution of suitable aquaculture areas
using \texttt{tmap}.

\begin{Shaded}
\begin{Highlighting}[]
\CommentTok{\# Rasterize eez regions}
\NormalTok{eez\_rast }\OtherTok{\textless{}{-}} \FunctionTok{rasterize}\NormalTok{(eez, avg\_sst\_depth\_suitable, }
                      \AttributeTok{field =} \StringTok{"rgn"}\NormalTok{) }\CommentTok{\# By region}
\CommentTok{\# Identify suitable cells in mask }
\NormalTok{suitable\_cells\_eez }\OtherTok{\textless{}{-}}  \FunctionTok{mask}\NormalTok{(avg\_sst\_depth\_suitable, eez)}

\CommentTok{\# Calculate cell areas (km\^{}2)}
\NormalTok{cell\_area }\OtherTok{\textless{}{-}} \FunctionTok{cellSize}\NormalTok{(suitable\_cells\_eez, }\AttributeTok{unit =} \StringTok{"km"}\NormalTok{)}

\CommentTok{\# Convert to sf object}
\NormalTok{eez\_sf }\OtherTok{\textless{}{-}} \FunctionTok{st\_as\_sf}\NormalTok{(eez) }\CommentTok{\# To have geometry}

\NormalTok{area\_eez }\OtherTok{\textless{}{-}} \FunctionTok{zonal}\NormalTok{(cell\_area }\SpecialCharTok{*}\NormalTok{ suitable\_cells\_eez, }\CommentTok{\# Identify area of suitable locations }
\NormalTok{                  eez\_rast, }\CommentTok{\# Rasterized eez}
                  \AttributeTok{fun =} \StringTok{"sum"}\NormalTok{, }\AttributeTok{na.rm =}\NormalTok{ T) }\SpecialCharTok{\%\textgreater{}\%}  \CommentTok{\# Sum areas of cells within each EEZ zone }
  \FunctionTok{rename}\NormalTok{(}\AttributeTok{suitable\_area\_km2 =}\NormalTok{ area) }\SpecialCharTok{\%\textgreater{}\%} \CommentTok{\# Rename for naming conventions}
  \FunctionTok{as.data.frame}\NormalTok{() }\SpecialCharTok{\%\textgreater{}\%} \CommentTok{\# Convert to data frame}
  \FunctionTok{left\_join}\NormalTok{(eez\_sf, }\AttributeTok{by =} \StringTok{"rgn"}\NormalTok{) }\CommentTok{\# Join on region}

\CommentTok{\# Convert back into sf so eez data (includes calculated suitable area) for mapping}
\NormalTok{area\_eez }\OtherTok{\textless{}{-}}\NormalTok{ area\_eez }\SpecialCharTok{\%\textgreater{}\%} \FunctionTok{st\_as\_sf}\NormalTok{()}
\end{Highlighting}
\end{Shaded}

\begin{Shaded}
\begin{Highlighting}[]
\CommentTok{\# Create map}
\NormalTok{oyster\_pref\_map }\OtherTok{\textless{}{-}} \FunctionTok{tm\_shape}\NormalTok{(area\_eez) }\SpecialCharTok{+}
  \FunctionTok{tm\_polygons}\NormalTok{(}
    \StringTok{"suitable\_area\_km2"}\NormalTok{, }\CommentTok{\# Color by suitable\_area\_km2 variable}
    \AttributeTok{palette =} \StringTok{"{-}mako"}\NormalTok{, }\CommentTok{\# Reverse blue scale}
    \AttributeTok{style =} \StringTok{"cont"}\NormalTok{, }\CommentTok{\# Continuous scale (styles referenced:}
    \CommentTok{\#https://r{-}tmap.github.io/tmap{-}book/visual{-}variables.html)}
    \AttributeTok{title =} \FunctionTok{expression}\NormalTok{(}
      \StringTok{"Suitable Area"}\SpecialCharTok{\textasciitilde{}}\StringTok{"("}\SpecialCharTok{\textasciitilde{}}\NormalTok{ km}\SpecialCharTok{\^{}}\DecValTok{2}\SpecialCharTok{\textasciitilde{}}\StringTok{")"}\NormalTok{) }\CommentTok{\# Rename legend title}
\NormalTok{  ) }\SpecialCharTok{+}  
  
  \FunctionTok{tm\_text}\NormalTok{(}\StringTok{"rgn"}\NormalTok{, }\CommentTok{\# Label by region}
          \AttributeTok{size =}\NormalTok{ .}\DecValTok{8}\NormalTok{, }\CommentTok{\# Adjust size}
          \AttributeTok{col =} \StringTok{"white"}\NormalTok{, }\CommentTok{\# Adjust text color}
          \AttributeTok{fontface =} \StringTok{"bold"}\NormalTok{,  }\CommentTok{\# Labels are bolded}
          \AttributeTok{xmod =} \SpecialCharTok{{-}}\NormalTok{.}\DecValTok{5}\NormalTok{) }\SpecialCharTok{+} \CommentTok{\# Adjust .5 from the left}
  
  \FunctionTok{tm\_layout}\NormalTok{( }\CommentTok{\# Center title outside bounding box}
    \AttributeTok{main.title =} \StringTok{"Marine Aquaculture Suitability for Oysters in West Coast EEZs"}\NormalTok{,}
     \AttributeTok{main.title.size =} \FloatTok{1.5}\NormalTok{, }\CommentTok{\# Adjust title size}
    \AttributeTok{legend.outside =} \ConstantTok{TRUE}\NormalTok{, }\CommentTok{\# Place legend outside map frame}
    \AttributeTok{legend.outside.position =} \StringTok{"right"}\NormalTok{, }\CommentTok{\# Place legend to right}
    \AttributeTok{component.autoscale =} \ConstantTok{FALSE}\NormalTok{, }\CommentTok{\# Disable autoscaling for title}
    \AttributeTok{outer.margins =} \FunctionTok{c}\NormalTok{(}\FloatTok{0.01}\NormalTok{, }\FloatTok{0.25}\NormalTok{, }\FloatTok{0.01}\NormalTok{, }\FloatTok{0.05}\NormalTok{) }\CommentTok{\# Manually adjust map frame}
\NormalTok{    ) }\SpecialCharTok{+} 
  
   \FunctionTok{tm\_scale\_bar}\NormalTok{( }\CommentTok{\# Add scale bar for scale}
     \AttributeTok{position =} \FunctionTok{c}\NormalTok{(}\SpecialCharTok{{-}}\NormalTok{.}\DecValTok{01}\NormalTok{, }\FloatTok{0.08}\NormalTok{),   }\CommentTok{\# Move 1\% from left and 8\% from bottom }
     \AttributeTok{breaks =} \FunctionTok{seq}\NormalTok{(}\DecValTok{0}\NormalTok{, }\DecValTok{500}\NormalTok{, }\DecValTok{150}\NormalTok{)) }\SpecialCharTok{+} \CommentTok{\# Establish scale bar ranges}

     \FunctionTok{tm\_compass}\NormalTok{( }\CommentTok{\# Add compass for orientation}
     \AttributeTok{type =} \StringTok{"4star"}\NormalTok{,        }
     \AttributeTok{position =} \FunctionTok{c}\NormalTok{(}\StringTok{"right"}\NormalTok{, }\StringTok{"top"}\NormalTok{)) }\SpecialCharTok{+} \CommentTok{\# Adjust position}
  \FunctionTok{tm\_basemap}\NormalTok{(}\StringTok{"Esri.OceanBasemap"}\NormalTok{) }\CommentTok{\# Ocean basemap}

\CommentTok{\# Save finalized map to figs}
\FunctionTok{tmap\_save}\NormalTok{(oyster\_pref\_map, }\AttributeTok{filename =} \StringTok{"images/oyster\_pref\_map.png"}\NormalTok{, }\AttributeTok{width =} \DecValTok{8}\NormalTok{, }\AttributeTok{height =} \DecValTok{10}\NormalTok{)}
\end{Highlighting}
\end{Shaded}

\includegraphics{images/oyster_pref_map.png} \textbf{Map 1. Finalized
map of suitable aquaculture areas within West Coast EEZs for oysters.}

The suitable area for harvesting oysters relative to the total area of
each EEZ region was also a topic of interest alongside visualizing
suitable aquaculture area within each EEZ. I created a table using
\texttt{kableExtra} to display both the total suitable area and the
proportion of each EEZ that is suitable for oyster aquaculture, enabling
comparison across regions independent of EEZ size.

\begin{Shaded}
\begin{Highlighting}[]
\CommentTok{\# Table with kableextra for prop of suitable areas to EEZ area}
\NormalTok{area\_eez }\SpecialCharTok{\%\textgreater{}\%}
\NormalTok{  st\_drop\_geometry }\SpecialCharTok{\%\textgreater{}\%}  \CommentTok{\# Drop geometry}
\NormalTok{  dplyr}\SpecialCharTok{::}\FunctionTok{select}\NormalTok{(}\AttributeTok{region =}\NormalTok{ rgn, }\CommentTok{\# Select region}
\NormalTok{         suitable\_area\_km2, }\CommentTok{\# Select suitable area}
         \AttributeTok{total\_area\_km2 =}\NormalTok{ area\_km2 }\CommentTok{\# Rename to total\_area}
\NormalTok{         ) }\SpecialCharTok{\%\textgreater{}\%}   
  \CommentTok{\# Update suitable\_area to be rounded to nearest hundreth}
  \FunctionTok{mutate}\NormalTok{(}\AttributeTok{suitable\_area\_km2 =} \FunctionTok{round}\NormalTok{(suitable\_area\_km2, }\DecValTok{2}\NormalTok{), }
         \CommentTok{\# Create new variable for prop of suitable area to total EEZ area}
         \AttributeTok{percent\_suitable =} \FunctionTok{round}\NormalTok{((suitable\_area\_km2 }\SpecialCharTok{/}\NormalTok{ total\_area\_km2) }\SpecialCharTok{*} \DecValTok{100}\NormalTok{, }
                                  \DecValTok{1}\NormalTok{)) }\SpecialCharTok{\%\textgreater{}\%} \CommentTok{\# Round to nearest tenth}
  
\NormalTok{  dplyr}\SpecialCharTok{::}\FunctionTok{select}\NormalTok{(}\SpecialCharTok{{-}}\NormalTok{total\_area\_km2) }\SpecialCharTok{\%\textgreater{}\%} \CommentTok{\# Deselect total\_area}
  
  \CommentTok{\# Rename columns}
  \FunctionTok{rename}\NormalTok{(}
    \StringTok{"Region"} \OtherTok{=}\NormalTok{ region,}
     \StringTok{"Suitable Area (km\^{}2)"} \OtherTok{=}\NormalTok{ suitable\_area\_km2, }
    \StringTok{"Proportion of Suitable Area in EEZ"} \OtherTok{=}\NormalTok{ percent\_suitable}
\NormalTok{    )  }\SpecialCharTok{\%\textgreater{}\%}

  \CommentTok{\# Enable title}
  \FunctionTok{kable}\NormalTok{(}\AttributeTok{caption =} \StringTok{"Amount of Suitable Areas by EEZ for Oyster Preferences"}\NormalTok{) }\SpecialCharTok{\%\textgreater{}\%} 
  \CommentTok{\# Allow table to be striped with highlight option}
  \FunctionTok{kable\_styling}\NormalTok{(}\AttributeTok{bootstrap\_options =} \FunctionTok{c}\NormalTok{(}\StringTok{"striped"}\NormalTok{, }\StringTok{"hover"}\NormalTok{),}
                \AttributeTok{full\_width =} \ConstantTok{FALSE}\NormalTok{, }\CommentTok{\# Disable full width }
                \AttributeTok{position =} \StringTok{"center"}\NormalTok{)  }\CommentTok{\# Center labels}
\end{Highlighting}
\end{Shaded}

\begin{longtable}[t]{lrr}
\caption{Amount of Suitable Areas by EEZ for Oyster Preferences}\\
\toprule
Region & Suitable Area (km\textasciicircum{}2) & Proportion of Suitable Area in EEZ\\
\midrule
Central California & 3656.82 & 1.8\\
Northern California & 194.13 & 0.1\\
Oregon & 1028.90 & 0.6\\
Southern California & 3062.20 & 1.5\\
Washington & 2435.93 & 3.6\\
\bottomrule
\end{longtable}

\hypertarget{part-2-generalized-function-of-aquaculture-for-species-preferences}{%
\subsection{\texorpdfstring{\textbf{Part 2: Generalized Function of
Aquaculture for Species
Preferences}}{Part 2: Generalized Function of Aquaculture for Species Preferences}}\label{part-2-generalized-function-of-aquaculture-for-species-preferences}}

The workflow observed in Part 1 is applicable for a specific species of
interest in generating a map of suitable areas for aquaculture within
West Coast EEZs based on species-specific temperature and depth ranges.
However, creating a reproducible workflow allows a user to recreate the
same visualization specific to a species of interest based on a
preferred species's temperature and depth range. As such, I created the
\texttt{species\_preference} function which applies the generalized
workflow (Part 1) to streamline identification of suitable zones based
on species-specific sea surface temperature (SST) and depth ranges for
any species of interest. The function takes minimum and maximum SST
values, minimum and maximum depth limits, and a species name as inputs
and returns a map of EEZ regions shaded by total suitable area.

It is important to note that \texttt{species\_preference} only considers
calculations for the specific inputs, such as reclassification and
average SST conversion. As a result, I reloaded all spatial objects and
verified shared CRSs to ensure all spatial objects were correctly
prepared for processing in the function.

\begin{Shaded}
\begin{Highlighting}[]
\CommentTok{\# Load in data again}

\CommentTok{\# West Coast EEZ}
\NormalTok{eez }\OtherTok{\textless{}{-}} \FunctionTok{vect}\NormalTok{(here}\SpecialCharTok{::}\FunctionTok{here}\NormalTok{(}\StringTok{"posts"}\NormalTok{, }\StringTok{"aquaculture{-}post"}\NormalTok{, }\StringTok{"data\_aqua"}\NormalTok{, }\StringTok{"wc\_regions\_clean.shp"}\NormalTok{))}

\CommentTok{\# Bathymetry raster}
\NormalTok{depth }\OtherTok{\textless{}{-}} \FunctionTok{rast}\NormalTok{(here}\SpecialCharTok{::}\FunctionTok{here}\NormalTok{(}\StringTok{"posts"}\NormalTok{, }\StringTok{"aquaculture{-}post"}\NormalTok{,}\StringTok{"data\_aqua"}\NormalTok{, }\StringTok{"depth.tif"}\NormalTok{))}

\CommentTok{\# Create a list of the tiff files for SST}
\NormalTok{sst\_files }\OtherTok{\textless{}{-}} \FunctionTok{list.files}\NormalTok{( }\CommentTok{\# Define file path}
  \AttributeTok{path =} \StringTok{"/Users/vedikashirtekar/Documents/MEDS/mysite/posts/aquaculture{-}post/data\_aqua"}
\NormalTok{,}
  \CommentTok{\# Pattern matches all tiff files starting with   "average\_annual"}
  \AttributeTok{pattern =} \StringTok{"average\_annual.*}\SpecialCharTok{\textbackslash{}\textbackslash{}}\StringTok{.tif$"}\NormalTok{,}
  \AttributeTok{full.names =}\NormalTok{ T }
\NormalTok{)}


\CommentTok{\# Stop running if character(0) returned}
\ControlFlowTok{if}\NormalTok{ (}\FunctionTok{length}\NormalTok{(sst\_files) }\SpecialCharTok{==} \DecValTok{0}\NormalTok{) \{}
  \FunctionTok{stop}\NormalTok{(}\StringTok{"No SST raster files found in directory."}\NormalTok{)}
\NormalTok{\}}

\CommentTok{\# Stack raster files}
\NormalTok{sst }\OtherTok{\textless{}{-}} \FunctionTok{rast}\NormalTok{(sst\_files)}



\CommentTok{\# Create list of spatial objects}
\NormalTok{spatial\_objects }\OtherTok{\textless{}{-}} \FunctionTok{list}\NormalTok{(eez, depth, sst)}

\CommentTok{\# Use eez\textquotesingle{}s CRS as reference}
\NormalTok{ref\_crs }\OtherTok{\textless{}{-}} \FunctionTok{st\_crs}\NormalTok{(spatial\_objects}\SpecialCharTok{$}\NormalTok{eez)}

\CommentTok{\# Check and transform each tile with if/else statements}
\ControlFlowTok{if}\NormalTok{ (}\FunctionTok{st\_crs}\NormalTok{(spatial\_objects}\SpecialCharTok{$}\NormalTok{depth) }\SpecialCharTok{!=}\NormalTok{ ref\_crs) \{}
  \FunctionTok{warning}\NormalTok{(}\StringTok{"depth CRS does not match. Transforming to match eez CRS."}\NormalTok{)}
\NormalTok{  spatial\_objects}\SpecialCharTok{$}\NormalTok{depth }\OtherTok{\textless{}{-}} \FunctionTok{st\_transform}\NormalTok{(spatial\_objects}\SpecialCharTok{$}\NormalTok{depth, ref\_crs)}
\NormalTok{\} }\ControlFlowTok{else}\NormalTok{ \{}
  \FunctionTok{message}\NormalTok{(}\StringTok{"depth CRS already matches eez CRS."}\NormalTok{)}
\NormalTok{\}}


\CommentTok{\# Check and transform each tile with if/else statements}
\ControlFlowTok{if}\NormalTok{ (}\FunctionTok{st\_crs}\NormalTok{(spatial\_objects}\SpecialCharTok{$}\NormalTok{sst) }\SpecialCharTok{!=}\NormalTok{ ref\_crs) \{}
  \FunctionTok{warning}\NormalTok{(}\StringTok{"sst CRS does not match. Transforming to match eez CRS."}\NormalTok{)}
\NormalTok{  spatial\_objects}\SpecialCharTok{$}\NormalTok{sst }\OtherTok{\textless{}{-}} \FunctionTok{st\_transform}\NormalTok{(spatial\_objects}\SpecialCharTok{$}\NormalTok{sst, ref\_crs)}
\NormalTok{\} }\ControlFlowTok{else}\NormalTok{ \{}
  \FunctionTok{message}\NormalTok{(}\StringTok{"sst CRS already matches eez CRS."}\NormalTok{)}
\NormalTok{\}}
\end{Highlighting}
\end{Shaded}

I also redefined the \texttt{multiply} function to support raster
multiplication within the \texttt{species\_preference} function when
used with \texttt{lapp()}.

\begin{Shaded}
\begin{Highlighting}[]
\CommentTok{\# Define multiply function for global raster multiplication}
\NormalTok{multiply }\OtherTok{\textless{}{-}} \ControlFlowTok{function}\NormalTok{(x,y)\{ }
\NormalTok{  multi\_raster }\OtherTok{\textless{}{-}}\NormalTok{ x}\SpecialCharTok{*}\NormalTok{y}
  \FunctionTok{return}\NormalTok{(multi\_raster)}
\NormalTok{  \}}
\end{Highlighting}
\end{Shaded}

The following code cells demonstrate the integrated workflow utilized in
the \texttt{species\_preference} function to generate a map of suitable
aquaculture areas within West Coast EEZs based on species-specific SST
and depth parameters. I called the function on the specific temperature
and depth range for the red abalone to replicate a map similar to that
of the suitable oyster aquaculture areas.

\begin{Shaded}
\begin{Highlighting}[]
\CommentTok{\# This function takes arguments:}
\CommentTok{\# minimum and maximum sea surface temperature}
\CommentTok{\# minimum and maximum depth}
\CommentTok{\# species name}

\NormalTok{species\_preference }\OtherTok{\textless{}{-}} \ControlFlowTok{function}\NormalTok{(min\_temp, max\_temp, min\_depth, max\_depth, species\_name)\{ }

\DocumentationTok{\#\#\# Assume files have been loaded in already with matching CRS checks}
  
\DocumentationTok{\#\#\#\_\_\_\_\_\_\_\_\_\_\_\_\_\_\_\_\_\_\_\_\_\_\_\_\_\_\_\_\_\_\_\_\_\_\_\_\_\_\_\_\_\_\_\_\_\_\_\_\_\_\_\_\_\_\_\_\_\_\_\_}
\DocumentationTok{\#\#\# Data processing}

\CommentTok{\# Reproject to match sst CRS}
\NormalTok{depth }\OtherTok{\textless{}{-}} \FunctionTok{project}\NormalTok{(depth, sst)}

\CommentTok{\# Calculate average SST among all rasters}
\NormalTok{avg\_sst }\OtherTok{\textless{}{-}} \FunctionTok{mean}\NormalTok{(sst)}

\CommentTok{\# Update avg\_sst in degrees Celsius}
\NormalTok{avg\_sst }\OtherTok{\textless{}{-}}\NormalTok{ avg\_sst }\SpecialCharTok{{-}} \FloatTok{273.15}

\CommentTok{\# Use crop() to crop depth to the extent of avg\_sst}
\NormalTok{depth\_sst\_crop }\OtherTok{\textless{}{-}} \FunctionTok{crop}\NormalTok{(depth, avg\_sst)}

\CommentTok{\# Resample with nearest neighbor method}
\NormalTok{depth\_sst\_crop }\OtherTok{\textless{}{-}} \FunctionTok{resample}\NormalTok{(depth\_sst\_crop, avg\_sst, }\AttributeTok{method =} \StringTok{"near"}\NormalTok{)}

\DocumentationTok{\#\#\#\_\_\_\_\_\_\_\_\_\_\_\_\_\_\_\_\_\_\_\_\_\_\_\_\_\_\_\_\_\_\_\_\_\_\_\_\_\_\_\_\_\_\_\_\_\_\_\_\_\_\_\_\_\_\_\_\_\_\_\_}
\DocumentationTok{\#\#\# Find suitable locations}

\CommentTok{\# Define reclass matrix for un/suitable SST}
\NormalTok{reclass\_matrix\_sst }\OtherTok{\textless{}{-}} \FunctionTok{matrix}\NormalTok{(}
  \FunctionTok{c}\NormalTok{(}\SpecialCharTok{{-}}\ConstantTok{Inf}\NormalTok{, min\_temp, }\DecValTok{0}\NormalTok{, }\CommentTok{\# Negative infinity (unbounded) to min\_temp}
\NormalTok{    min\_temp, max\_temp, }\DecValTok{1}\NormalTok{, }\CommentTok{\# min{-}max temp assigned 1}
\NormalTok{    max\_temp, }\ConstantTok{Inf}\NormalTok{, }\DecValTok{0}\NormalTok{), }\CommentTok{\# max\_temp to infinity (unbounded) assigned 0}
  \AttributeTok{ncol =} \DecValTok{3}\NormalTok{, }\CommentTok{\# Create three columns}
  \AttributeTok{byrow =}\NormalTok{ T }\CommentTok{\# Fill by row }
\NormalTok{)}

\CommentTok{\# Assign reclassified values to avg\_sst}
\NormalTok{avg\_sst\_reclass }\OtherTok{\textless{}{-}} \FunctionTok{classify}\NormalTok{(avg\_sst, }\AttributeTok{rcl =}\NormalTok{ reclass\_matrix\_sst)}

\CommentTok{\# Define reclass matrix for un/suitable depth}
\NormalTok{reclass\_matrix\_depth }\OtherTok{\textless{}{-}} \FunctionTok{matrix}\NormalTok{(}
  \FunctionTok{c}\NormalTok{(}\SpecialCharTok{{-}}\ConstantTok{Inf}\NormalTok{, min\_depth, }\DecValTok{0}\NormalTok{, }\CommentTok{\# Negative infinity (unbounded) to min\_temp (below sea level)}
\NormalTok{    min\_depth, max\_depth, }\DecValTok{1}\NormalTok{, }\CommentTok{\# min{-}max depth assigned 1}
\NormalTok{    max\_depth, }\ConstantTok{Inf}\NormalTok{, }\DecValTok{0}\NormalTok{), }\CommentTok{\# \textgreater{} max\_depth assigned unsuitable (0)}
  \AttributeTok{ncol =} \DecValTok{3}\NormalTok{,}\CommentTok{\# Create three columns}
  \AttributeTok{byrow =}\NormalTok{ T }\CommentTok{\# Fill by row  }
\NormalTok{)}

\CommentTok{\# Assign reclassified values to depth}
\NormalTok{depth\_reclass }\OtherTok{\textless{}{-}} \FunctionTok{classify}\NormalTok{(depth\_sst\_crop, }\AttributeTok{rcl =}\NormalTok{ reclass\_matrix\_depth)}

\CommentTok{\# Return suitable (1) and unsuitable (0) cells }
\NormalTok{avg\_sst\_depth }\OtherTok{\textless{}{-}} \FunctionTok{lapp}\NormalTok{(}
  \AttributeTok{x =} \FunctionTok{c}\NormalTok{(avg\_sst\_reclass,depth\_reclass), }\CommentTok{\# Stack rasters}
  \AttributeTok{fun =}\NormalTok{ multiply) }\CommentTok{\# Apply multiplication function }


\DocumentationTok{\#\#\#\_\_\_\_\_\_\_\_\_\_\_\_\_\_\_\_\_\_\_\_\_\_\_\_\_\_\_\_\_\_\_\_\_\_\_\_\_\_\_\_\_\_\_\_\_\_\_\_\_\_\_\_\_\_\_\_\_\_\_\_}
\DocumentationTok{\#\#\# Determine most suitable locations within EEZs}

\CommentTok{\# Project}
\NormalTok{eez }\OtherTok{\textless{}{-}} \FunctionTok{project}\NormalTok{(eez, avg\_sst\_depth)}

\CommentTok{\# Select suitable areas}
\NormalTok{avg\_sst\_depth\_suitable }\OtherTok{\textless{}{-}} \FunctionTok{ifel}\NormalTok{(avg\_sst\_depth }\SpecialCharTok{==} \DecValTok{0}\NormalTok{, }
                               \ConstantTok{NA}\NormalTok{, }\CommentTok{\# Replace with NA}
                               \DecValTok{1}\NormalTok{) }\CommentTok{\# Otherwise assign "1"}

\CommentTok{\# Rasterize eez regions}
\NormalTok{eez\_rast }\OtherTok{\textless{}{-}} \FunctionTok{rasterize}\NormalTok{(eez, avg\_sst\_depth\_suitable, }
                      \AttributeTok{field =} \StringTok{"rgn"}\NormalTok{) }\CommentTok{\# By region}

\CommentTok{\# Identify suitable cells in mask }
\NormalTok{suitable\_cells\_eez }\OtherTok{\textless{}{-}}  \FunctionTok{mask}\NormalTok{(avg\_sst\_depth\_suitable, eez)}

\CommentTok{\# Calculate cell areas (km\^{}2)}
\NormalTok{cell\_area }\OtherTok{\textless{}{-}} \FunctionTok{cellSize}\NormalTok{(suitable\_cells\_eez, }\AttributeTok{unit =} \StringTok{"km"}\NormalTok{)}

\CommentTok{\# Convert to sf object}
\NormalTok{eez\_sf }\OtherTok{\textless{}{-}} \FunctionTok{st\_as\_sf}\NormalTok{(eez) }\CommentTok{\# To have geometry}

\NormalTok{area\_eez }\OtherTok{\textless{}{-}} \FunctionTok{zonal}\NormalTok{(cell\_area }\SpecialCharTok{*}\NormalTok{ suitable\_cells\_eez, }\CommentTok{\# Identify area of suitable locations }
\NormalTok{                  eez\_rast, }\CommentTok{\# Rasterized eez}
                  \AttributeTok{fun =} \StringTok{"sum"}\NormalTok{, }\AttributeTok{na.rm =}\NormalTok{ T) }\SpecialCharTok{\%\textgreater{}\%}  \CommentTok{\# Sum areas of cells within each EEZ zone }
  \FunctionTok{rename}\NormalTok{(}\AttributeTok{suitable\_area\_km2 =}\NormalTok{ area) }\SpecialCharTok{\%\textgreater{}\%} \CommentTok{\# Rename for naming conventions}
  \FunctionTok{as.data.frame}\NormalTok{() }\SpecialCharTok{\%\textgreater{}\%} \CommentTok{\# Convert to data frame}
  \FunctionTok{left\_join}\NormalTok{(eez\_sf, }\AttributeTok{by =} \StringTok{"rgn"}\NormalTok{) }\CommentTok{\# Join on region}

\CommentTok{\# Convert back into sf so eez data (includes calculated suitable area) for mapping}
\NormalTok{area\_eez }\OtherTok{\textless{}{-}}\NormalTok{ area\_eez }\SpecialCharTok{\%\textgreater{}\%} \FunctionTok{st\_as\_sf}\NormalTok{()}

\DocumentationTok{\#\#\#\_\_\_\_\_\_\_\_\_\_\_\_\_\_\_\_\_\_\_\_\_\_\_\_\_\_\_\_\_\_\_\_\_\_\_\_\_\_\_\_\_\_\_\_\_\_\_\_\_\_\_\_\_\_\_\_\_\_\_\_}
\DocumentationTok{\#\#\# Create map of suitable areas within EEZs}
\FunctionTok{tm\_shape}\NormalTok{(area\_eez) }\SpecialCharTok{+}
  \FunctionTok{tm\_polygons}\NormalTok{(}
    \StringTok{"suitable\_area\_km2"}\NormalTok{, }\CommentTok{\# Color by suitable\_area\_km2 variable}
    \AttributeTok{palette =} \StringTok{"{-}mako"}\NormalTok{, }\CommentTok{\# Reverse blue scale}
    \AttributeTok{style =} \StringTok{"cont"}\NormalTok{, }\CommentTok{\# Continuous scale }
    \AttributeTok{title =} \FunctionTok{expression}\NormalTok{(}
      \StringTok{"Suitable Area"}\SpecialCharTok{\textasciitilde{}}\StringTok{"("}\SpecialCharTok{\textasciitilde{}}\NormalTok{ km}\SpecialCharTok{\^{}}\DecValTok{2}\SpecialCharTok{\textasciitilde{}}\StringTok{")"}\NormalTok{) }\CommentTok{\#Rename legend title}
\NormalTok{  ) }\SpecialCharTok{+}  
  \FunctionTok{tm\_text}\NormalTok{(}\StringTok{"rgn"}\NormalTok{, }\CommentTok{\# Label by region}
          \AttributeTok{size =}\NormalTok{ .}\DecValTok{8}\NormalTok{, }\CommentTok{\# Adjust size}
          \AttributeTok{col =} \StringTok{"white"}\NormalTok{, }\CommentTok{\# Adjust text color}
          \AttributeTok{fontface =} \StringTok{"bold"}\NormalTok{,  }\CommentTok{\# Labels are bolded}
          \AttributeTok{xmod =} \SpecialCharTok{{-}}\NormalTok{.}\DecValTok{5}\NormalTok{) }\SpecialCharTok{+} \CommentTok{\# Adjust .5 from the left}
  
  \FunctionTok{tm\_layout}\NormalTok{( }\CommentTok{\# Center title outside bounding box}
    \AttributeTok{main.title =} \FunctionTok{paste}\NormalTok{(}\StringTok{"Marine Aquaculture Suitability for"}\NormalTok{,}
\NormalTok{                        species\_name, }\CommentTok{\# Include species name in title}
                       \StringTok{"in West Coast EEZs"}\NormalTok{),}
     \AttributeTok{main.title.size =} \FloatTok{1.5}\NormalTok{, }\CommentTok{\# Adjust title size}
    \AttributeTok{legend.outside =} \ConstantTok{TRUE}\NormalTok{, }\CommentTok{\# Place legend outside map frame}
    \AttributeTok{legend.outside.position =} \StringTok{"right"}\NormalTok{, }\CommentTok{\# Place legend to right}
    \AttributeTok{component.autoscale =} \ConstantTok{FALSE}\NormalTok{, }\CommentTok{\# Disable autoscaling for title}
    \AttributeTok{outer.margins =} \FunctionTok{c}\NormalTok{(}\FloatTok{0.01}\NormalTok{, }\FloatTok{0.25}\NormalTok{, }\FloatTok{0.01}\NormalTok{, }\FloatTok{0.05}\NormalTok{) }\CommentTok{\# Manually adjust map frame}
\NormalTok{    ) }\SpecialCharTok{+} 
  
   \FunctionTok{tm\_scale\_bar}\NormalTok{( }\CommentTok{\# Add scale bar for scale}
     \AttributeTok{position =} \FunctionTok{c}\NormalTok{(}\SpecialCharTok{{-}}\NormalTok{.}\DecValTok{01}\NormalTok{, }\FloatTok{0.08}\NormalTok{),   }\CommentTok{\# Move 1\% from left and 8\% from bottom }
     \AttributeTok{breaks =} \FunctionTok{seq}\NormalTok{(}\DecValTok{0}\NormalTok{, }\DecValTok{500}\NormalTok{, }\DecValTok{150}\NormalTok{)) }\SpecialCharTok{+} \CommentTok{\# Establish scale bar ranges}

     \FunctionTok{tm\_compass}\NormalTok{( }\CommentTok{\# Add compass for orientation}
     \AttributeTok{type =} \StringTok{"4star"}\NormalTok{,        }
     \AttributeTok{position =} \FunctionTok{c}\NormalTok{(}\StringTok{"right"}\NormalTok{, }\StringTok{"top"}\NormalTok{)) }\SpecialCharTok{+} \CommentTok{\# Adjust position}
  \FunctionTok{tm\_basemap}\NormalTok{(}\StringTok{"Esri.OceanBasemap"}\NormalTok{) }\CommentTok{\# Ocean basemap}
\NormalTok{\}}

\CommentTok{\# Output:}
\CommentTok{\# Map of EEZ regions colored by amount of suitable area}
\end{Highlighting}
\end{Shaded}

\begin{Shaded}
\begin{Highlighting}[]
\CommentTok{\# Call function for Red Abalone and preferred temp/depth range}
\NormalTok{species\_pref\_map }\OtherTok{\textless{}{-}} \FunctionTok{species\_preference}\NormalTok{(}\AttributeTok{min\_temp =} \DecValTok{8}\NormalTok{, }
                                       \AttributeTok{max\_temp =} \DecValTok{18}\NormalTok{, }
                                       \AttributeTok{min\_depth =} \SpecialCharTok{{-}}\DecValTok{25}\NormalTok{, }
                                       \AttributeTok{max\_depth =} \DecValTok{0}\NormalTok{, }
                                       \AttributeTok{species\_name =} \StringTok{"Red Abalone"}\NormalTok{)}

\CommentTok{\# Save finalized map to figs}
\FunctionTok{tmap\_save}\NormalTok{(species\_pref\_map, }\AttributeTok{filename =} \StringTok{"images/species\_pref\_map.png"}\NormalTok{, }\AttributeTok{width =} \DecValTok{8}\NormalTok{, }\AttributeTok{height =} \DecValTok{10}\NormalTok{)}
\end{Highlighting}
\end{Shaded}

\includegraphics{images/species_pref_map.png} \textbf{Map 2. Finalized
map of suitable aquaculture areas within West Coast EEZs for red
abalone.}

\hypertarget{reflection}{%
\subsection{Reflection}\label{reflection}}

Central California is the most suitable EEZ with about 3,656.82
km\textsuperscript{2} of suitable area while Northern California is the
least suitable EEZ with about 194.13 km\textsuperscript{2} of suitable
area. Central California likely offers the most favorable SST and depth
conditions for oyster cultivation. Notably, the Washington EEZ has a
larger suitable area (1,028.90 km²) than Northern California despite its
colder location, indicating a higher proportion of habitat appropriate
for mollusk aquaculture. Central and Southern California also show
relatively high suitability. Oregon's EEZ contains substantial suitable
area as well, though less than most other regions.

The red abalone, a critically endangered marine gastropod, is a
commercially valuable species with strong potential for aquaculture. The
Washington EEZ appears especially promising for red abalone cultivation,
with more than 1,500 km² of suitable area, likely due to the colder
water temperatures preferred by the species. Interestingly, Central and
Southern California also contain substantial suitable areas that exceed
those of Northern California, despite being further south and having
generally warmer conditions; differences in coastal upwelling and
nutrient availability may help explain this pattern. Red abalone
aquaculture also represents a model for sustainable seafood production.
For instance, the Monterey Abalone Company in Central California raises
abalone in both marine and land-based systems with minimal use of
chemicals and antibiotics (Bailey, 2015). Overall, red abalone farms
highlight opportunities for environmentally responsible aquaculture
while supporting production in the most suitable West Coast EEZs,
particularly in Washington, Central California, and Northern California.

Future Considerations

\hypertarget{references}{%
\subsection{References}\label{references}}

{[}1{]} Alleway, H. K., Gillies, C. L., Bishop, M. J., Gentry, R. R.,
Theuerkauf, S. J., \& Jones, R. (2018). \emph{The ecosystem services of
marine aquaculture: Valuing benefits to people and nature}. BioScience,
69(1), 59--68. \url{https://doi.org/10.1093/biosci/biy137}

{[}2{]} \emph{Gridded bathymetry data}. (n.d.). GEBCO. Retrieved
November 30, 2025, from
\url{https://www.gebco.net/data-products/gridded-bathymetry-data\#area}

{[}3{]} \emph{Haliotis rufescens,~Red abalone: Fisheries.} (n.d.).
Retrieved November 30, 2025, from
\url{https://www.sealifebase.ca/summary/Haliotis-rufescens.html}

{[}4{]} \emph{Marine regions.} (n.d.). Retrieved November 28, 2025, from
\url{https://www.marineregions.org/eez.php}

\emph{Monterey Bay Abalone farm shows what sustainable aquaculture can
be like.} (n.d.). Earth Island Journal. Retrieved November 29, 2025,
from

\url{https://www.earthisland.org/journal/index.php/articles/entry/monterey_bay_abalone_farm_shows_what_sustainable_aquaculture_can_be_like/}

{[}5{]} \emph{NOAA coral reef watch daily 5km satellite coral bleaching
heat stress SST anomaly product} (version 3.1). (n.d.). Retrieved
November 29, 2025, from
\url{https://coralreefwatch.noaa.gov/product/5km/index_5km_ssta.php}

{[}6{]} Oliver, R. (2025). \emph{Homework assignment 4.} Retrieved
November 28, 2025, from
\url{https://eds-223-geospatial.github.io/assignments/HW4.html\#fnref1}

{[}7{]} Wright, A., Moody, C., \& Gross, J. (2025). \emph{Composition of
California's aquaculture industry and surveying its disease challenges
and management strategies.} Aquaculture Reports, 42, 102799. Retrieved
November 29, 2025, from

\url{https://doi.org/10.1016/j.aqrep.2025.102799}



\end{document}
